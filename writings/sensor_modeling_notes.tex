\documentclass[a4paper, 11pt]{article}
\usepackage{fullpage}
\setlength{\parindent}{0in}

\usepackage{setspace}
\onehalfspacing

\usepackage{amsmath}
\usepackage{amssymb}
\usepackage{graphicx}
\usepackage{caption}
\usepackage{subcaption}
\usepackage{comment}
\usepackage{ifthen,version}
\usepackage{color}
\DeclareMathOperator{\tr}{tr}
\newboolean{include-notes}
%\setboolean{include-notes}{true}
\newcommand{\atnote}[1]{\ifthenelse{\boolean{include-notes}}%
 {\textcolor{red}{\textbf{AT: #1}}}{}}

\author{}
\title{Sensor modeling notes}
\date{}

\begin{document}
\maketitle

\section{The structure of the paper}
\subsection{Motivation for the work}

\subsection{What are the obvious factors to model? And how are these to be tested?}
\subsubsection*{Geometric factors}
\begin{itemize}
\item
From \cite{neato_sensor}, given $q = \dfrac{fs}{x}$,
  \begin{align*}
    \Delta q = \frac{q}{fs}\Delta fs + \frac{q^2}{fs}\Delta x - \frac{q^2}{s\sin^2 \beta}\Delta \beta
  \end{align*}
Try figuring out the numerical values of these coefficients. If the neato unit was manufactured with the same parameters as in the paper, $fs = 800mm, s = 50mm, \beta \approx 82^\circ$, Can the contribution due these various factors be separated out? $\Delta x$ is the only term on the right with a life of its own. What is measured is $d = \frac{q}{\sin \beta}$, so
  \begin{align*}
    \Delta d = d\left(\frac{\Delta fs}{fs} - \cot \beta\Delta \beta\right) + d^2\left(\frac{\sin \beta\Delta x}{fs} - \frac{\Delta \beta}{s\sin \beta}\right)
  \end{align*}
\item The laser beam may not lie in the intended horizontal plane. How will this be modeled?
\item Laser zero may not be aligned with robot heading zero. Test this by putting robot in a square cage.
\end{itemize}

\subsubsection*{Error in centroid reporting}
This is an expansion of the $\Delta x$ term that occurs in geometric factors. A number of factors could contribute to this error. 
\begin{itemize}
\item Angle of incidence affects how much energy is reflected.
\item Ambient conditions and material reflectivity.
\end{itemize}

\subsubsection*{Lens distortion}
Hypothesis is that the effect will become important both at near and far ranges. 

\subsubsection*{Temperature effects}
The number of returns seems to reduce with time. Design test for this.

\subsubsection*{Spinning-reporting errors}
Is the laser spinning uniformly? Or is the reporting back well-timed?

\subsubsection*{Motion blur effects}
Due to the robot's motion. This can be modeled directly.

\subsubsection*{Change of coordinate frames}
Look at same errors but under different coordinate frames: they may become simpler to explain.

\subsection{Data collection}
What is notation convention? Raw data in the form of $R\times M\times N\times T$. $R$ is number of range data, $M$ is orientation, $N$ is measurement set, $T$ is return index. 

\subsection{Non-trivial non-idealities}
Present evidence that there are non-idealities that couldn't account for. The following examples were directly observed. But what about non-idealities which are never observed, because of the sheer volume of data? How are anomalies to be detected? 
  \begin{itemize}
    \item Choose a level of granularity: return, measurement set, etc. Consider data as elements of a set at this level of granularity.
    \item Choose a set of parameters to describe an element of the data set. What are these parameters? I don't know, but one way is to throw in a bunch of parameters for which there is the slightest indication of dependency, then identify from the distributions which of these are redundant.
    \item Given all the elements, there now exists a distribution of the parameters. 
    \item The outliers of this distribution perhaps represent non-idealities. 
  \end{itemize}

\subsubsection*{Directional bias}
Is the laser system isotropic? Are there systematic errors from some returns? Systematic bias should turn up as the mean.

\subsubsection*{Whiteout}
For some certain measurement sets, very little data. Why is that? From $\mathtt{processed\_data2.mat}$, lookat $\mathtt{errors\{i\}}$, where $i = 1, 180, 200, 240$.

\subsubsection*{Anomaly detection}
These are found completely by observation
  \begin{itemize}
  \item For certain return indices, there are $0$ data points. They are contiguous sets of indices.
  \end{itemize}

\subsection{Application of work}
Work backwards from goal of the paper. Want to show that for some algorithms, the data generated by simulator is as bad as that of the real world, and so tweaking parameters can be done in simulation, and the algorithm will work in reality as well. This is the claim we want to make. So think of algorithms which will benefit: don't choose a weak algorithm just because it strengthens the cause of this paper. Choose a robust, relevant example. Let this goal direct the decisions on which factors to model, leave out, in what depth et cetera.


%useful skeletons
\begin{comment}
  \begin{align*}

  \end{align*}

  \begin{itemize}
    \item
  \end{itemize}
\end{comment}

\begin{thebibliography}{9}

\bibitem{neato_sensor}
Kurt Konolige, Joseph Augenbraun, Nick Donaldson, Charles Fiebig, and Pankaj Shah. A Low-Cost Laser Distance Sensor. ICRA 2008.

\end{thebibliography}

\end{document}

